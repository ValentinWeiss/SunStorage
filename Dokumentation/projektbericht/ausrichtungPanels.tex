\chapterauthor{Felix Wagner}
Die Ausrichtung der Solarpanele erfolgt über die vom Haupt-ESP berechneten Azimut und Elevation Werte.
Diese repräsentieren die Position der Sonne am Himmel. 
Wenn davon ausgegangen wird, dass der SunStorage Aufbau stets auf einer ebenen Fläche liegt, dann können die Azimut- und Elevationwerte direkt auf die Postionen der Rotations- und Kippservos übertragen werden. 
Somit würde sich eine direkte Ausrichtung zur Sonne ergeben und die Effektivität der Solarpanele wäre maximiert.
Dieser Fall ist jedoch im praktischen Betrieb nicht immer gegeben.

\subsubsection{Schieflage des Systems}
Steht der Aufbau schief, so ist die Trennung der Aufgabengebiete des Rotieren und Kippens der Solarpanele nicht mehr gegeben.
Somit können die Azimut und Elevation nicht mehr direkt für die Stellung der Servomotoren verwendet werden.
Kippt man etwa den kompletten Aufbau auf eine Seite, so sind die Aufgaben der Motoren komplett umgekehrt.
Für alle Orientierungen dazwischen übernehmen die Motoren Rotation und Kippen gemeinsam.

\paragraph{Beschleunigungswerte}
Um die Lage des SunStorage Systems zu ermitteln, muss die Erdbeschleunigung anhand von drei Achsen gemessen werden. 
Dies erfolgt über die Daten des Beschleunigungssensors.

\paragraph{Interpretation der Beschleunigungswerte}
Betrachtet man die Beschleunigungswerte der drei Achsen als Vektor im 3-dimensionalen Raum, 
so spannt dieser Vektor eine Ebene auf. 
Dieser Vektor muss nun an der z-Achse im Koordinatensystem gespiegelt werden. 
Damit lässt sich die physikalische Lage von SunStorage im Koordinatensystem abbilden.
Dieser Vektor wird im weiteren Verlauf $ V_{SunStorage} $ genannt.

\paragraph{Darstellen der Sonne im Koordinatensystem}
Nachdem nun SunStorage im Koordinatensystem abgebildet ist, muss die gewünschte Sonnenposition mittels eines Vektors hinzugefügt werden.
Wie bereits bekannt wird die Sonnenposition über die Werte Azimut und Elevation angegeben. 
Dabei handelt es sich um Winkel. 
Diese können mithilfe der trigonometrischen Funktionen in einem 3-dimensionalen Vektor übersetzt werden.
Hierfür wird die Berechnung in vier verschiedene Fälle aufgeteilt, welche die vier Quadranten eines 2-dimensionalen Koordinatensystems darstellen. 
\begin{itemize}
    \item I Quadrant: $x = 1 y = \tan(Azimuth)$
    \item II Quadrant: $x = -1 y = -\tan(Azimuth)$
    \item III Quadrant: $x = -1 y = -\tan(Azimuth)$
    \item IV Quadrant: $x = -\tan(Azimuth) y = -1$
\end{itemize}
Anschließend wird der Vektor aus dem Punkt (x,y) und dem Ursprung normalisiert, um mit dem später berechneten z-Wert die Sonnenposition darzustellen.\\
Die z-Koordinate wird anschließend über den Tangens des Elevationwertes bestimmt. 

$$ z = \tan(Elevation) $$

Nun stellen x, y und z den Sonnenvektor $ V_{Sonne} $ dar.\\
Ist der Elevationswinkel 90°, so wird $ V_{Sonne} $ auf (0,0,1) gesetzt, da in diesem Fall der Azimutwert vernachlässigbar ist.
Nun kann die Sonnenposition im Koordinatensystem dargestellt werden.

\paragraph{Solarpanele darstellen}
Um die Orientierung der Solarpanele im Koordinatensystem darzustellen, wird eine Ebene konstruiert, welche die Panele senkrecht in ihrer Mitte trennt.
Diese wird über die Berechnung des Orthogonalvektors der z-Achse und $ V_{SunStorage} $ bestimmt:
\[
\begin{pmatrix}
    0\\
    0\\
    1\\
\end{pmatrix}
\times
V_{SunStorage}
=
V_{ortho}
\]
Aus dem Vektor $ V_{ortho} $ lässt sich die Koordinatenform als Ebene erstellen.
\[V_{ortho}x \cdot x + V_{ortho}y \cdot y + V_{ortho}z \cdot z = 0\] 
Nun kann diese Ebene um die Achse von $ V_{SunStorage} $ rotiert werden.
Dies wird über die allgemeine Drehmatrix umgesetzt.
\[
R_n(\alpha)
=
\begin{pmatrix}
    n^2_1(1-\cos\alpha)+\cos\alpha & n_1n_2(1-\cos \alpha)-n_3\sin\alpha & n_1n_3(1-\cos\alpha)+n_2\sin\alpha\\
    n_2n_1(1-\cos\alpha)+n_3\sin\alpha & n^2_2(1-\cos\alpha)+\cos\alpha & n_2n_3(1-\cos\alpha)-n_1\sin\alpha\\
    n_3n_1(1-\cos\alpha)-n_2\sin\alpha & n_3n_2(1-\cos\alpha)+n_1\sin\alpha & n^2_3(1-\cos\alpha)+\cos\alpha\\
\end{pmatrix}    
\]
An Stelle von $ n = (n_1, n_2, n_3)^T $ wird $ V_{SunStorage}^T $ eingesetzt.
Dies entspricht der Bewegung des Rotationsservomotors.
Anschließend kann die Rotation der Ebene gesucht werden, für die $ V_{Sonne} $ eingesetzt
in  die Koordinatenform der Ebene des Normalenvektors $ V_{ortho} $ den geringsten Wert hat. %TODO der satz doof
Damit ist die Bestimmung der Stellung des Rotationsmotors beendet, da dies bedeutet, 
dass $ V_{Sonne} $ in der Ebene von  $ V_{ortho} $ liegt.
Anschließend erfolgt die Berechnung der Stellung des Kippmotors.
Dafür wird ein weiterer Orthogonalvektor $ V_{opt} $ erzeugt. 
Dieser liegt zwischen der z-Achse und $ V_{ortho} $. 
Die Stellung des Kippmotors kann somit durch die Berechnung des Winkels zwischen $ V_{opt} $ und $ V_{Sonne} $ bestimmt werden:
\[\alpha = \arccos(\frac{V_{opt} \cdot V_{Sonne}}{\left| V_{opt} \right| \cdot \left| V_{Sonne} \right|})\]

\subsubsection{Schieflage des Kompass}
Der Kompass liefert in Abhängigkeit seiner Ausrichtung unterschiedliche Werte.
In diesem Kapitel wird die notwendige Korrektur beschrieben.
\paragraph{Nordausrichtung}
\chapterauthor{Felix Wagner}
Die Bestimmung der Orientierung zu Norden erfolgt über die Winkelberechnung zwischen der 
y-Achse und der Geraden des Koordinatenursprungs und des aktuellen (x, y)-Messpunkts. 
Dieser Winkel wird mittels der Tangensfunktion ermittelt. Siehe Abbildung \autoref{fig:circlecenter}.
Die C Standardbibiliothek $math.h$ liefert hierfür die Funktion $atan2f$, 
welche den Winkel als float Datentyp zurückgibt 
und die vier verschiedenen Quadranten des Koordinatensystems berücksichtigt.

\paragraph{Schieflagenkorrektur}
\chapterauthor{Felix Wagner}
Es ist nicht immer gegeben, dass der Kompass perfekt gerade liegt. 
Deshalb müssen die Sensorwerte unter Berücksichtigung des vom Beschleunigungssensor 
gestelltem Orientierungsvektor angepasst werden. 
Dies erfolgt über die sogenannte Vektorprojektion. 
Dabei wird über folgende Formel \[\vec{a_{projected}} = \frac{\vec{a} \circ \vec{b}}{\vec{b} \circ \vec{b}} \cdot \vec{b} \]
der Vektor aus den Kompasswerten auf die Orientierung des Vektors der Beschleunigungswerte projeziert. 
Anschließend werden die x- und y-Werte des projezierten Vektors von denen des Kompassvektors subtrahiert. 
Somit sind die Kompasswerte bereit für die oben besprochene Weiterverarbeitung.

\subsubsection{Verbesserter Algorithmus}
Der implementierte Algorithmus nutzt für seine Berechnung viele Multiplikationen und trigonometrische Funktionen.
Dies erzeugt einen großen Rechenaufwand und ist im Dauerbetrieb nicht optimal.
Dies wirkt sich aufgrund der großen zeitlichen Abstände der Funktionsaufrufe kaum negativ auf die Funktionsweise von SunStorage aus.
Im Folgenden wird eine effizientere Formel erklärt.\\
Die Berechnung von $ V_{SunStorage} $  und $ V_{Sonne} $ erfolgt analog zum ursprünglichen Algorithmus.
Anstatt einer Ebene für die Solarpanele eine Ebene über den Vektor $ V_{SunStorage} $ in Koordinatenform erzeugt.
\[V_{SunStorage}x \cdot x + V_{SunStorage}y \cdot y + V_{SunStorage}z \cdot z = 0\] 
Anschließend wird die Spitze von $ V_{Sonne} $ als Ausgangspunkt für eine Gerade $ G_{Sonne} $ mit dem Richtungsvektor $ V_{SunStorage} $ generiert.
Der Schnittpunkt von $ G_{Sonne} $ mit der Ebene von SunStorage stellt mit dem Ursprung des Koordinatensystems den Vektor $ V_{opt} $ dar.
Aus $ V_{opt} $ kann wiederum die Stellung für Rotation und Kippen der Motoren analog zum vorherigen Ansatz berechnet werden.

\subsubsection{Bestimmung der Abweichung zum Norden}
\chapterauthor{Felix Wagner}
Die Bestimmung der Orientierung zu Norden erfolgt über die Winkelberechnung zwischen der 
y-Achse, der Geraden des Koordinatenursprungs und des aktuellen (x, y)-Messpunkts. 
Dieser Winkel wird mittels der Tangensfunktion ermittelt. (siehe \autoref{fig:circlecenter})
Die C Standardbibliothek $math.h$ liefert hierfür die Funktion $atan2f$, 
welche den Winkel als float Datentyp zurückgibt und dabei die vier verschiedenen Quadranten des Koordinatensystems berücksichtigt.

\paragraph{Schieflagenkorrektur}
Es ist nicht immer gegeben, dass der Kompass perfekt gerade liegt. 
Deshalb müssen die Sensorwerte unter Berücksichtigung des Orientierungsvektors, die der Beschleunigungssensor liefert, angepasst werden. 
Dies erfolgt über eine Vektorprojektion.
Mithilfe der folgenden Formel wird der Vektor aus den Kompasswerten auf die Orientierung des Vektors der Beschleunigungswerte projiziert.
\[\vec{a_{projected}} = \frac{\vec{a} \circ \vec{b}}{\vec{b} \circ \vec{b}} \cdot \vec{b} \]
 
Anschließend werden die x- und y-Werte des projizierten Vektors von den Werten des Kompassvektors subtrahiert. 
Somit sind die Kompasswerte bereit für die in Weiterverarbeitung. \autocite{compasstilt}