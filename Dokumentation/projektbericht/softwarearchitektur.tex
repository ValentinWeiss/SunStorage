\subsection{Softwarearchitektur}
\chapterauthor{Johannes Treske}

Die Software des Projektes wurde auf Grundlage des ESP-IDF Frameworks von Espressif geschrieben.
Zur Wahl stand zum einen das ESP-IDF Framework und zum anderen das Arduino Framework.
Das Arduino Framework bietet bereits viele vorgefertigte Bibliotheken.
Es enthält beispielsweise Funktionen zur Ansteuerung verschiedener Sensoren, was die Komplexität des eigenen Codes stark vereinfacht hätte.
Jedoch hat das Arduino Framework auch einige Nachteile, wie einen größeren Speicherverbrauch oder einen limitierten FreeRTOS Support.
Dadurch ist unsere Wahl auf das, auch in der Industrie häufig eingesetzte, ESP-IDF Framework gefallen.
Dieses bietet vollständigen FreeRTOS Support und stellt unter anderem Treiber für alle Funktionalitäten der ESP32 Plattform bereit.

\subsubsection{FreeRTOS}
\chapterauthor{Johannes Treske}

FreeRTOS ist ein Open Source Real-Time Operating System für Mikrocontroller und Mikroprozessoren.
Für das Projekt relevante Funktionen von FreeRTOS sind:
\begin{itemize}
    \item Task Scheduling mit preemptiv Scheduling Verfahren
    \item Erstellen von Tasks mit Prioritäten
    \item Bereitstellen von weiteren Funktionen im Zusammenhang mit Tasks, wie zum Beispiel Delay Funktionen
    \item Erstellen und Nutzen von Mutexen
\end{itemize}

\subsubsection{Aufbau der Software}
\chapterauthor{Johannes Treske}

Die Software des Projektes ist in zwei Teile aufgeteilt.
Des Web-Frontend ist der Teil der Software, der auf Anfrage über WLAN an den Client ausgeliefert wird.
Anschließend wird dieser auf Client-Seite ausgeführt.
Dieser Teil der Software wird komprimiert auf dem Flash-Speicher des Haupt-ESP gespeichert.
In \autoref{Frontend} wird der Aufbau der Webseite genau beschrieben.
Der zweite Teil der Software ist das Backend, welches in Software für Haupt- und Neben-ESP aufgeteilt ist.

Als Programmiersprache wird für die Software auf den ESP32 C verwendet.
Die Software besteht aus einzelnen Tasks, die mithilfe des FreeRTOS Schedulers nebenläufig ausgeführt werden.
Um die Software für beide ESP32 in einem Projekt entwickeln zu können wird der Code mit den build flags \textit{-DTX} für den Haupt-ESP und \textit{-DRX} für den Neben-ESP kompiliert.
Dadurch ist es möglich im Code, Abschnitte die nur für einen der beiden Controller bestimmt sind, mit
\begin{lstlisting}[language=C, frame=none]
#ifdef TX
    // code
#endif
\end{lstlisting}
beziehungsweise
\begin{lstlisting}[language=C, frame=none]
#ifdef RX
    // code
#endif
\end{lstlisting}
abzugrenzen.
So wird nur der Programmcode erzeugt, der für den jeweiligen Controller relevant ist.
Alles was nicht abgegrenzt ist, wird immer kompiliert.

\paragraph{Task \textit{app\_main}}
\chapterauthor{Johannes Treske}
\textit{app\_main} ist die Funktion, die nach dem Starten des Programms von FreeRTOS als Task initialisiert und aufgerufen wird.
Zu diesem Zeitpunkt ist dies der einzige initialisierte Task.
Da \textit{app\_main} sowohl für den Haupt-ESP als auch für den Neben-ESP kompiliert werden muss, werden die Bereiche für den Haupt- und Neben-ESP, wie oben beschrieben, innerhalb der Funktion angegrenzt.

Die Aufgaben von \textit{app\_main} sind:
\begin{enumerate}
    \item \textbf{Initialisieren der Module}\newline
          Die Initialisierung der Module ist in eine eigene Funktion \textit{globalInit} ausgelagert, um eine bessere Übersicht zu wahren.
          Es ist wichtig, alle Module an einem Ort zu initialisieren, da einige Module von mehreren Tasks verwendet werden. 
          Dadurch wird eine Mehrfachinitialisierung verhindert.
          Bei kritischen Modulen führt das Fehlschlagen der Initialisierung zu einem Reset des Systems.
          Als kritische Module werden die Module betrachtet, die zur Überwachung des Akkus und für das Laden notwendig sind.
          Einige Module sind nur für den Haupt-ESP oder nur für den Neben-ESP relevant und werden entsprechend abgegrenzt.
          Der Erfolg der Initialisierung eines Moduls wird in einer globalen Variable pro Modul festgehalten.
          Dadurch können Tasks vor der Nutzung feststellen, ob das entsprechende Modul korrekt initialisiert wurde.
    \item \textbf{Erstellen der Tasks}\newline
          Das Erstellen der Tasks erfolgt mithilfe der FreeRTOS Funktion \textit{xTaskCreate}.
          Angegeben wird die C-Funktion, die den Task umsetzt, sowie die zur Verfügung gestellte Stack-Größe und die Taskpriorität.
          Wir haben für alle Tasks die Idle-Priorität gewählt, da es keinen Task gibt, der eine höhere Priorität erhalten sollte.
          Die Stack-Größe haben wir an die tatsächlich benötigte Stack-Größe angepasst, waren jedoch großzügig, da der ESP32 über ausreichend Arbeitsspeicher für unsere Anwendung verfügt.
          Sobald ein Task erstellt wurde, erhält er automatisch vom Scheduler Rechenzeit.
    \item \textbf{Return}\newline
          \textit{app\_main} ist der einzige Task, der enden darf.
          Alle anderen Tasks müssen endlos laufen.
\end{enumerate}

\paragraph{Task \textit{adcReadTask}}
\chapterauthor{Lukas Eigenstetter}
Der \textit{adcReadTask} läuft auf dem Haupt-ESP.
Die Aufgaben sind:
\begin{enumerate}
    \item \textbf{Auslesen der ADC-Pins}\newline
          Die ausgelesenen Werte sind die Spannung der höheren und der niedrigeren Batteriezelle, die Spannung am Ladegerät-Shunt, die Spannung am System-Shunt und die Spannung am Coulomb-Counter.
    \item \textbf{Anwendung der Formeln zur Bestimmung der tatsächlichen Werte}\newline
          Die Formeln sind Teil von \autoref{akkuCircuit}.
    \item \textbf{Bilden von Mittelwerten und Abspeichern der Werte für andere Tasks}\newline
          Das verwendete Verfahren wird in \autoref{akkuCircuit} beschrieben.
\end{enumerate}

\paragraph{Task \textit{batteryChargerTask}}
\chapterauthor{Lukas Eigenstetter}
Der \textit{batteryChargerTask} läuft auf dem Haupt-ESP.
Dieser Task dient zur Ansteuerung des Ladegerätes.
Er setzt die in \autoref{Akkuladegerät} beschriebene State Machine um.
Falls das Ladegerät aktiv ist, wird das PWM zur Ladestromsteuerung angepasst.
Außerdem liest er den Temperatursensor des Barometers aus, um eine Überhitzung zu erkennen und führt Cell-Balancing durch.

\paragraph{Task \textit{socCalibrationTask}}
\chapterauthor{Lukas Eigenstetter}
Dieser Task läuft auf dem Haupt-ESP.
Er dient dazu, den Ladezustandswert des Akkus neu zu kalibrieren.
Das verwendete Verfahren wird in \autoref{Ladezustandsabschätzung} genauer beschrieben.\\
Zur Messung werden das Ladegerät und der Buck-Converter abgeschaltet.
Die Wiederanschaltung der Komponenten kann dazu führen, dass extern herbeigeführte Veränderungen überschrieben werden.
Dieses Lost-Update wurde bis zur Abgabe nicht behoben.

\paragraph{Task \textit{storeReadingsTask}}
\chapterauthor{Johannes Treske}
Der \textit{storeReadingsTask} läuft auf dem Haupt-ESP.
Dieser Task dient dazu, Messwerte zu einem bestimmten Zeitpunkt aufzunehmen und in der Datenbank zu speichern.
Die Werte aus der Datenbank können dann über das Webinterface graphisch dargestellt werden.
Seine Aufgaben sind:
\begin{enumerate}
    \item \textbf{Auslesen der Messwerte}
          Die Messwerte, die für die Datenbank relevant sind, werden von den entsprechenden Modulen ausgelesen.
          Die Werte eines Moduls werden nur dann ausgelesen, wenn es erfolgreich initialisiert wurde.
          Andernfalls wird das Modul übersprungen und in die entsprechenden Spalten in der Datenbank wird \emph{NULL} eingetragen.
          \emph{NULL} wird auch dann eingetragen, wenn der Messwert nicht verfügbar ist, so zum Beispiel, wenn das GPS Modul keine Satelliten findet.
          Als ID dient ein Timestamp nach dem folgenden Schema:
          \begin{center}
              \textbf{YYYYMMDDhhmmss}
          \end{center}
          Dieser wird aus den Daten des GPS Moduls ermittelt.
    \item \textbf{Speichern des Datenpaketes in der Datenbank}
\end{enumerate}

\paragraph{Task \textit{sunAlignTask}}
\chapterauthor{Felix Wagner}
Im \textit{sunAlignTask} werden alle nötigen Daten für die Berechnung der aktuellen Sonnenposition gesammelt.
Primär sind hierfür die Längen- und Breitengrade, welche vom GPS-Modul bereitgestellt werden, sowie das aktuelle Datum und Uhrzeit notwendig.
Daraus kann über den im Webfrontend beschriebenen Algorithmus die aktuelle Sonnenposition in Form von Azimut und Elevation berechnet werden.
\newline
Außerdem werden noch weitere Daten gesammelt, welche an den Neben-ESP übergeben werden.
\begin{itemize}
    \item Kompass: Abweichung vom magnetischen Nordpol
    \item GPS-Sensor: Abweichung vom wahren Nordpol (Magnetic Variation)
    \item Beschleunigungssensor: Lage des Apparats
\end{itemize}

Die Beschleunigungswerte werden in die State-Variablen geladen.
Nachdem alle Daten gesammelt wurden, wird der aktuelle Betriebsmodus der Panel-Ausrichtung über die Statevariable abgefragt.
Ist die automatische Ausrichtung aktiv, so wird die aktuelle Abweichung vom Nordpol zum Azimut addiert.
Azimut und Elevation  werden dann in die State-Variable geschrieben.
Im manuellen Modus können die Azimut-, Elevation- und Kompasswerte mithilfe von Eingabefeldern im Webinterface eingestellt werden.
Diese werden anschließend ebenfalls als Azimut und Elevation in die State-Variablen geladen. In \autoref{tableStateSunAlign} werden alle nötigen State-Variablen aufgelistet:

\begin{table}[H]
    \begin{center}
        \begin{tabular}{|l|l|l|}
            \hline
            Klasse & Name                  & Beschreibung                                        \\
            \Xhline{3\arrayrulewidth}
            positioning & manualMode & Einstellung des automatischen Modus im Webinterface       \\
            \hline
            positioning & azimuth           & Azimut aus Webinterface                            \\
            \hline
            positioning & elevation         & Elevation aus Webinterface                         \\
            \hline
            positioning & angleToNorth      & Nordabweichung aus Webinterface                    \\
            \hline
            esp2 & azimuth        & Übergebene Azimut an Neben-ESP                               \\
            \hline
            esp2 & elevation      & Übergebene Elevation an Neben-ESP                            \\
            \hline
            esp2 & (x,y,z)Gyro           & Weiterleitung der Beschleunigungswerte an Neben-ESP   \\
            \hline
        \end{tabular}
        \caption{State-Variablen für sunAlignTask}
        \label{tableStateSunAlign}
    \end{center}
\end{table}


\paragraph{Task \textit{usbChargerTask}}
\chapterauthor{Lukas Eigenstetter}
Der \textit{usbChargerTask} läuft auf dem Haupt-ESP und dient zur Anpassung der Ausgänge der USB-Ladeschnittstelle.
Die DAC-PINs werden auf eine bestimmte Spannung eingestellt.
Somit kann dem zu ladenden Endgerät die maximale Ladeleistung signalisiert werden.
Dies wir in \autoref{usbSchnittstelle} detailliert erklärt.


\paragraph{Task \textit{writeSystemStateTask}}
\chapterauthor{Lukas Eigenstetter}
Mithilfe des \textit{writeSystemStateTask} speichert der Haupt-ESP den Systemzustand regelmäßig.
Der Zustand liegt als JSON Objekt vor.
Das Objekt wird als Datei \textbf{state.txt} in das Hauptverzeichnis der SD-Card abgelegt.
Der Systemzustand wird in \autoref{Systemzustand} detailliert erklärt.

Falls eine Veränderung des Zustands seit der letzten Ausführung des Tasks vorliegt,
wird der State an den Neben-ESP gesendet.

\paragraph{Task \textit{statsCalcTask}}
\chapterauthor{Matthias Unterrainer}


\paragraph{Task \textit{displayTask}}
\chapterauthor{Matthias Unterrainer}


\paragraph{Task \textit{displayButtonsTask}}
\chapterauthor{Matthias Unterrainer}


\paragraph{Task \textit{espCommReceiveTask}}
\chapterauthor{Johannes Treske}
Der \textit{espCommReceiveTask} läuft auf dem Neben-ESP.
Verändert sich der System-State verändert, so wird ein flag gesetzt, das dem \emph{writeSystemStateTask} signalisiert, dass der State übertragen werden soll. 
Dann sendet der Haupt-ESP den State im JSON-Format an den Neben-ESP.
Dabei wird das JSON Objekt als eine einzige Zeile gesendet.
Der \textit{espCommReceiveTask} dient dazu, den gesendeten State zu empfangen und auf dem Neben-ESP für die anderen Tasks abzuspeichern.

\paragraph{Task \textit{panelAlignTask}}
\chapterauthor{Felix Wagner}
Der Neben-ESP führt außerdem noch den \textit{panelAlignTask} aus.
Dabei werden die vom Haupt-ESP in die State-Variablen geladenen Beschleunigungswerte, sowie die Azimut und Elevation geladen (siehe \autoref{tableStatePanelAlign}).
Anschließend werden die benötigten Stellungen der Servomotoren unter Berücksichtigung der Lage des SunStorage-Aufbaus berechnet.
Mit diesen Werten erfolgt die Stellung der Servomotoren.

\begin{table}[H]
    \begin{center}
        \begin{tabular}{|l|l|l|}
            \hline
            Klasse & Name                  & Beschreibung                                        \\
            \Xhline{3\arrayrulewidth}
            esp2 & azimuth        & Übergebene Azimut an Neben-ESP                               \\
            \hline
            esp2 & elevation      & Übergebene Elevation an Neben-ESP                            \\
            \hline
            esp2 & (x,y,z)Gyro           & Weiterleitung der Beschleunigungswerte an Neben-ESP   \\
            \hline
        \end{tabular}
        \caption{State-Variablen für panelAlignTask}
        \label{tableStatePanelAlign}
    \end{center}
\end{table}





