\vspace{1em}
\begin{table}[!hp]
    \begin{center}
        \begin{tabular}{|p{0.8cm}|p{6cm}|p{0.8cm}|p{8cm}|} \hline           
        \textbf{Typ} & \textbf{Aufgabe} & \textbf{Std.} & \textbf{Bemerkung} \\ \hline

        \multicolumn{4}{|l|}{Woche 1}                                                            \\ \hline
        S           &  Materiallien im Labor suchen & 2 & - \\ \hline
        \multicolumn{4}{|l|}{Woche 2}                                                            \\ \hline
        \multicolumn{4}{|l|}{Woche 3}                                                            \\ \hline
        I & LCD Display Ansteuerung Implementierung & 12 & - \\ \hline
        \multicolumn{4}{|l|}{Woche 4}                                                            \\ \hline
        B & Timing Probleme beim LCD Display beheben & 3 & - \\ \hline        
        \multicolumn{4}{|l|}{Woche 5}                                                            \\ \hline
        I & Implementierung LCD Menu & 8 & - \\ \hline
        \multicolumn{4}{|l|}{Woche 6}                                                            \\ \hline
        B & Überarbeitung des LCD Menus & 5 & Anpassen der Anzeigewerte und Einbindung der Display Konfiguration \\ 
        SF & Coulomb-Counter - Konzeptentwicklung & 10 & Inklusive Recherche der grundlegenden Funktionsweise \\ \hline
        \multicolumn{4}{|l|}{Woche 7}                                                            \\ \hline
        S & Coulomb-Counter - Schaltplan entwickeln und Testen & 12 & Erster Aufbau auf Steckbrett \\ \hline
        \multicolumn{4}{|l|}{Woche 8}                                                            \\ \hline
        S & Aufbau der Buttons und Coulomb-Counters & 2 & - \\
        I & Implementierung DCF77 & 22 & - \\ \hline
        \multicolumn{4}{|l|}{Woche 9}                                                            \\ \hline
        S & Besorgen der Materiallien für Panel Aufbau & 1 & - \\
        IS & Implementierung der Software für den Coulomb-Counter und Tests & 5 & - \\
        S & Bau der Anlage & 3 & Aufbau der Panele \\  \hline
        \multicolumn{4}{|l|}{Woche 10}                                                           \\ \hline
        I & Implementierung der Nachtabschaltung & 15 & Hauptsächlich Recherche bezüglich des Algorithmus zu Berechnung der Sonnenauf-/untergangszeiten \\ 
        B & Coulomb-Counter: Bugfixes & 8 & Änderung des Aufbaus um Messungen zu verbessern \\ 
        SB & Neubauen des Coulomb-Counters & 1 & - \\\hline           
    \end{tabular}
    \end{center}
    \label{tab:overviewMatthias1}
    \caption{Wochenübersicht und Stunden Matthias Unterrainer Teil 1}
        I: Implementierung, D: Dokumentation, H: Hilfe/Code-Review, S: Sonstiges\\
        F: Neuerung/Feature, B: Bug oder Verbesserung\\
        
\end{table}

\vspace{1em}

\begin{table}[!hp]
    \begin{center}
        \begin{tabular}{|p{0.8cm}|p{6cm}|p{0.8cm}|p{8cm}|} \hline
            \textbf{Typ} & \textbf{Aufgabe}                & \textbf{Std.} & \textbf{Bemerkung}                                                               \\ \hline    
            \multicolumn{4}{|l|}{Woche 11}                                                           \\ \hline
            SB & Umbau der Nachtabschaltung um zu neuen Taskstruktur zu passen & 2 & - \\
            S   & Bau der Anlage & 6 & Oberbau  \\  \hline
            \multicolumn{4}{|l|}{Woche 12}                                                           \\ \hline
            B & Überarbeitung DCF77 & 4 & Versuch das DCF77 Modul in unserem Aufbau nutzbar zu machen \\ 
            B & Beheben von Problemen beim Einbau von Coulomb-Counter & 3 & - \\ \hline          
            \multicolumn{4}{|l|}{Woche 13}                                                           \\ \hline
            S   & Bau der Anlage & 6 & Unterbau \\
            SB  & Lötstellen am Neben-ESP korrigieren & 3 & - \\
            H & Helfen bei Problemen mit dem Aufbau & 2 & - \\
            SB & LCD Display Aufbau fertigstellen & 1 & - \\ \hline     
            \multicolumn{4}{|l|}{Woche 14 bis 17}                                                    \\ \hline      
            S            & Aufbau reparieren               & 1             & Unterer Servomotor hat Servohorn kaputt gemacht \\
            BS & Integration des LCD Displays & 4 & - \\
            BS & Integration des Coulomb-Counters & 5 & - \\
            DF & Doku fertigstellen                     & 8 & - \\ \hline
            \Xhline{3\arrayrulewidth}
            & Gesamt & 154 & \\ \hline
        \end{tabular}
        \end{center}
        \label{tab:overviewMatthias2}
        \caption{Wochenübersicht und Stunden Matthias Unterrainer Teil 2}
            I: Implementierung, D: Dokumentation, H: Hilfe/Code-Review, S: Sonstiges\\
            F: Neuerung/Feature, B: Bug oder Verbesserung\\
\end{table}
\vspace{1em}

\newpage