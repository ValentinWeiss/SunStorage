\vspace{1em}
\begin{table}[!hp]
    \begin{center}
    \begin{tabular}{|p{0.8cm}|p{6cm}|p{0.8cm}|p{8cm}|} \hline
        \textbf{Typ} & \textbf{Aufgabe} & \textbf{Std.} & \textbf{Bemerkung} \\ \hline
        \multicolumn{4}{|l|}{Woche 1}                                                            \\ \hline
        DF           & Erstellung der Dokutemplate & 1.5           & -                  \\ 
        S            & Materialliste und Laborgegenstände sammeln & 2 & -\\
		DF           & Requirements Ausformulieren, Gantt         & 8 & -\\ \hline
        \multicolumn{4}{|l|}{Woche 2}                                                            \\ \hline
        DB& Requirements Verfeinern                    & 1 & -\\
        SF & Ladegerät Konzept und erster Schaltplan   & 5 & -\\ \hline
        \multicolumn{4}{|l|}{Woche 3}                                                            \\ \hline
        SF & Schaltplan Erstellen und Simulieren      & 8 & Inklusive Einarbeitung in das Simulationsprogramm \\ \hline
        \multicolumn{4}{|l|}{Woche 4}                                                            \\ \hline
        SF & Cell Balancing Planen und Simulieren     & 4 & -\\
        SD & Schaltung auf Board Planen und Zeichnen  & 15 & auch Woche 5\\ \hline
        \multicolumn{4}{|l|}{Woche 5}                                                            \\ \hline
        SF & Stromversorgung auf Pinboard Testen      & 4 & -\\
        SF & Ladegerät auf Pinboard Testen            & 5 & -\\ \hline
        \multicolumn{4}{|l|}{Woche 6}                                                            \\ \hline
        IF & ADCs Ansteuern                          & 5 & -\\
        IF & Ladegerät Algorithmus implementieren   & 15 & Inklusive Konzept, UML und Bugfixes \\ \hline
        \multicolumn{4}{|l|}{Woche 7}                                                            \\ \hline
        SF & Stromversorgung Löten                   & 5 & -\\
        SF & Ladegerät Löten                         & 10 & auch Woche 8\\ \hline
        \multicolumn{4}{|l|}{Woche 8}                                                            \\ \hline
        IF & State Implementieren                    & 10& Inklusive spätere Bugfixes und Erweiterungen \\
        SF & Buck Converter Schaltung Planen        & 5 & -\\
        SB & Lötstellen verbessern                   & 3 & auch weitere kleine Lötarbeiten (z.b. Kabel)\\
        IB & ADCs Kalibrieren                        & 1 & - \\ \hline
        \multicolumn{4}{|l|}{Woche 9 Teil 1}                                                \\ \hline
		DF & API.md Erstellen & 3 & Infos zu Backend-Web-Frontend - Mapping\\
        SF & Cell Balancing Löten                    & 8 & -\\  \hline
        \end{tabular}
    \label{tab:overviewLukas1}
    \end{center}
    \caption{Wochenübersicht und Stunden Lukas Eigenstetter Teil 1}
\end{table}

\vspace{1em}
\begin{table}[!hp]
    \begin{center}
    \begin{tabular}{|p{0.8cm}|p{6cm}|p{0.8cm}|p{8cm}|} \hline
        \textbf{Typ} & \textbf{Aufgabe} & \textbf{Std.} & \textbf{Bemerkung} \\ \hline
        \multicolumn{4}{|l|}{Woche 9 Teil 2}                                                \\ \hline        
        IH & Sqlite Installieren                     & 2 & erfolglos, wird nur für ältere Versionen unterstützt \\
        IF & Onboard Temperatursensor lesen          & 3 & erfolglos, wird bei verwendetem ESP nicht unterstützt\\
        DF & eigene Tasks definieren                 & 1.5&- \\ \hline
        \multicolumn{4}{|l|}{Woche 10}                                                           \\ \hline
        IF & Soc Konzept und Implementierung         & 5 & -\\
        SF & Buck Converter Verlöten                 & 4 & -\\ \hline
        \multicolumn{4}{|l|}{Woche 11}                                                           \\ \hline
        SB & Projektbericht maintainen               & 4 & über den ganzen restlichen Projektverlauf \\
        SF & Brett mit Schaltung aufbauen            & 12 & auch Woche 13, (inkl. Sensoren verlöten) \\
        IH & Code Review Matthias                    & 2 & Display Tasks\\ \hline
        \multicolumn{4}{|l|}{Woche 12}                                                           \\ \hline
        IH & Code Review Sun Align                   & 10& inklusive Verifikation des Verhaltens, auch Woche 13\\
        IH & CSV Code überprüfen                     & 3 & - \\
        IB & Build in RX und TX splitten            & 2 & -\\
        IB & Mittelwert für ADC Werte               & 2 & -\\ \hline
        \multicolumn{4}{|l|}{Woche 13}                                                           \\ \hline
        SB & Lötstellen korrigieren                 & 3 & Das Kupfer der zweiten Platine löst sich vom Brett \\
        DB & Doku schreiben                          & 2 & Doku für Schaltplan begonnen \\
        HB & Servo Ansteuerung mittels MCPWM        & 2 & Felix helfen\\
        DF & Eigene Präsentation vorberieten        & 4 & - \\
        DB & Präsentation ausbessern      & 2 & Einheitliche Formatierung, Rechtschreibfehler\\
        IB & Ausrichtealgorithmus         & 10 & Systemtests und Code Review                 \\ \hline
        \multicolumn{4}{|l|}{Woche 14 bis 17}                                                           \\ \hline
        DF & Pin-Layout mit KiCat zeichnen          & 15 & - \\
        IB & Servohorn fixen                        & 6 & Alu Servohorn, selbst geschweißtes Servohorn (erfolglos) \\
        DF & Doku fertigstellen                     & 15 & - \\
        DH & Doku Hilfe                & 3 & Fehler, einheitlicher Style\\        
        DF & Messdaten und Drehmechanismus aufnehmen & 2 & - \\ 
        DF & Fotos des Aufbau & 3 & Inklusive Doku der Sensorhalterung \\
        IB & mehrere Bugfixes & 6 & Ladegerät, CSV und Sun align \\ 
        \Xhline{3\arrayrulewidth}
        & Gesamt & 242 & \\ \hline
    \end{tabular}
    \end{center}
    \label{tab:overviewLukas2}
    \caption{Wochenübersicht und Stunden Lukas Eigenstetter Teil 2}
    I: Implementierung, D: Dokumentation, H: Hilfe/Code-Review, S: Sonstiges\\
        F: Neuerung/Feature, B: Bug oder Verbesserung
\end{table}
\vspace{1em}

\newpage